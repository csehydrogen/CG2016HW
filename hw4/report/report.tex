\documentclass{article}

\usepackage{kotex}
\usepackage{booktabs}
\usepackage{hyperref}
\usepackage{amsmath}

\begin{document}

\title{Programming Assignment \#4}
\author{김희훈}
\maketitle

\section{요약}

본 과제에서는 depth ordering을 이용한 반투명 물체 렌더링, 다양한 광원이 있는 상황에서 Phong illumination model을 이용한 서로 다른 특성을 가진 표면 렌더링을 구현했습니다.
추가 구현으로서 모든 물체 반투명 렌더링, BSP Tree, BSP 과정에서 나눠진 표면들을 확인하기 위한 테두리 렌더링 기능을 넣었습니다.

\section{컴파일 및 실행}

WebGL을 지원하는 최신 브라우저와 npm이 시스템에 설치되어 있어야 합니다.
쉘에서 다음 명령어들을 수행하여 컴파일합니다.

\begin{enumerate}
  \item cd /path/to/src
  \item npm install
  \item webpack
\end{enumerate}

index.html을 브라우저에서 실행하면 됩니다.
또는 \url{http://csehydrogen.github.io/CG2016HW/hw4/}에서 실행해 볼 수 있습니다.
최신 Chrome에서 실행을 확인하였습니다.

\section{조작법}

뷰를 마우스 드래그로 회전, wasd로 translate, rt로 dolly in/out, fg로 zoom in/out 할 수 있습니다.
렌더링을 위해서는 먼저 model칸에 데이터를 입력하거나 밑의 버튼을 눌러 사전에 정의된 모델을 자동 입력해야합니다.
그리고 density(자연수)를 지정한 후 apply를 누르면 렌더링 됩니다.
모든 물체 반투명화를 원한다면 apply를 누르기 이전에 translucent를 체크해 주세요.
wire, mesh 체크박스를 통해 렌더링 여부를 조절할 수 있습니다.
adjust view 버튼은 모델이 중앙에 오도록 view를 초기 위치로 조절해 줍니다.

\section{반투명 물체의 렌더링}

z-Buffer 만으로 반투명 물체의 렌더링을 흉내내려면 gl의 DEPTH\_TEST 대신 BLEND를 켠 후 깊은 순서대로 렌더링해야 합니다.
이번 숙제에서는 물체가 고정되어 있고 뷰만 바뀌는 상황이므로 BSP Tree를 구현하면 O(n)만에 depth ordering을 수행할 수 있습니다.
문제의 단순화를 위해 모든 표면을 삼각형 primitive만을 이용해 렌더링 하였습니다.

\subsection{BSP Tree construct}

BSP Tree의 생성은 BSP.js 안의 클래스 BSPTree의 생성자가 담당합니다.
이 생성자는 삼각형의 목록을 받아 다음의 알고리즘을 수행합니다.

\begin{enumerate}
  \item
    가장 큰 삼각형 $T_i$를 찾습니다.
    이는 분할되는 표면의 갯수를 줄이는 효과적인 휴리스틱입니다.
  \item
    모든 삼각형 $T_j$를 $T_i$가 포함된 평면으로 분할합니다.
    $T_j$는 분할 되지 않거나, 두 개의 삼각형으로 분할되거나, 삼각형과 사각형으로 분할됩니다.
    마지막 경우에서 사각형을 두 개의 삼각형으로 분할한다면, $T_j$는 최종적으로 최대 3개의 삼각형으로 분할됩니다.
  \item
    $T_i$의 법선 벡터를 기준으로 +방향에 있는지, 평면에 정확히 속하는지, -방향에 있는지를 판단하여 분할된 삼각형을 각각 리스트 $r$, $m$, $l$에 나누어 담습니다.
  \item
    리스트 $m$은 현재 노드에 저장하고, 리스트 $r$이 비어있지 않다면 $r$로 생성자를 재귀호출하여 서브트리를 만들어서 오른쪽 자식으로 삼습니다.
    $l$도 같은 방식으로 처리하여 왼쪽 자식을 만듭니다.
\end{enumerate}

\subsection{Depth ordering}

BSP Tree가 생성되어 있으면, 매 프레임마다 BSPTree.depthOrder을 호출하여 깊이 순으로 정렬된 삼각형 리스트를 넘겨받습니다.
이 함수는 먼저 현재 카메라 좌표가 $m$의 첫번째 삼각형이 속하는 평면을 기준으로 양의 방향에 있는지 음의 방향에 있는지 판단합니다.
만약 양의 방향에 있다면, 음의 방향에 있는 삼각형들, 즉 왼쪽 자식에 속한 삼각형을 먼저 렌더링 해야 되므로 왼쪽 자식을 먼저 방문하는 중위 순회를 수행합니다.
음의 방향이라면 반대 방향의 중위 순회를 수행합니다.
순회하면서 각 노드의 리스트 $m$을 차례대로 연결하면 깊이 순으로 정렬된 리스트가 됩니다.

\section{다양한 물질 표면 렌더링}

Phong illumination model을 사용하였으며 파라미터를 조정하여 다른 물질의 느낌을 내고자 하였습니다. 사용한 식은 다음과 같습니다.

\begin{equation*}
  I=k_aI_a+\Sigma_l\left(k_dI_l(N \cdot L)+k_sI_l(R \cdot V)^n\right)
\end{equation*}

$k_a$는 ambient 계수, $k_d$는 diffuse 계수, $k_s$는 specular 계수, $n$은 shininess, $I_a$는 ambient light의 밝기, $I_l$은 light source들의 밝기, $N$은 법선 벡터, $L$은 광원 벡터, $V$는 뷰 벡터, $R$은 빛이 반사되는 방향 벡터입니다.
$k_a$, $k_d$, $k_s$를 삼각형마다 설정해놓고, fragment shader에서 I를 계산하여 색 벡터와 곱하면 최종 색을 구할 수 있습니다.
계산 과정은 shader.frag에서 확인할 수 있습니다.

시연을 위해 큐브 5개와 점광원 5개를 사용하였습니다.
큐브의 각 면에 사용한 파라미터와 광원의 정보는 \autoref{fig:cube}와 \autoref{fig:light}를 참고해 주세요.
+z 방향의 점광원이 흰색이므로 색깔의 영향을 받지 않고 material을 판단하고자 할때는 큐브의 정면을 보는게 좋습니다.

\begin{figure}[ht]
  \centering
  \begin{tabular}{c c c c}
    위치 & material & 색깔 & 파라미터 \\
    \midrule
    좌 & metal & white (1.0, 1.0, 1.0) & (0.1, 0.1, 0.8, 32) \\
    중앙 & plastic & blue (0.4, 0.4, 1.0) & (0.2, 0.4, 0.4, 4) \\
    우 & mirror & white (1.0, 1.0, 1.0) & (0.0, 0.0, 1.0, 64) \\
    하 & rubber & grey (0.3, 0.3, 0.3) & (0.9, 0.1, 0.0, 1) \\
    상 & ruby & red (1.0, 0.2, 0.3) & (0.2, 0.0, 0.8, 16) \\
  \end{tabular}
  \caption{
    큐브 5개에 대한 자세한 정보.
    위치는 adjust view 이후 정면에서 바라봤을 때 기준.
    색깔은 (r, g, b), 파라미터는 ($k_a$, $k_d$, $k_s$, $n$).
    \label{fig:cube}
  }
\end{figure}

\begin{figure}[ht]
  \centering
  \begin{tabular}{c c c c}
    번호 & 위치 & 색깔 \\
    \midrule
    ambient & - & white (1, 1, 1) \\
    0 & (0, 0, d) & white (1, 1, 1) \\
    1 & (d, d, -d) & yellow (1, 1, 0) \\
    2 & (-d, d, -d) & red (1, 0, 0) \\
    3 & (-d, -d, -d) & green (0, 1, 0) \\
    4 & (d, -d, -d) & blue (0, 0, 1) \\
  \end{tabular}
  \caption{
    광원 5개에 대한 자세한 정보.
    위치는 (x, y, z), 색깔은 (r, g, b).
    d는 모델의 정점 중 원점에서 가장 먼 점까지의 거리로, 모델 크기에 따라 광원까지의 거리도 비례해서 커지도록 하였음.
    \label{fig:light}
  }
\end{figure}

\section{추가 구현}

\begin{itemize}
  \item Depth ordering에 BSP Tree를 구현하여 사용하였고, 시연 및 디버깅 목적으로 BSP 과정에서 잘려진 평면들을 보기 위해 테두리 렌더링 옵션을 추가하였습니다.
  \item 반투명 렌더링이 이 과제에만 적용되도록 ad-hoc하게 짜여진게 아니라, 보다 일반적인 복잡한 도형에도 적용될 수 있다는 점을 보이기 위해서 모든 물체를 반투명으로 렌더링 하는 옵션을 추가하였습니다.
\end{itemize}

\end{document}
