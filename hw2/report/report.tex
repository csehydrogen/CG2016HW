\documentclass[11pt]{article}

\usepackage{kotex}
\usepackage{hyperref}
\usepackage{amsmath}

\begin{document}

\title{Programming Assignment \#2}
\author{김희훈}
\maketitle

\section{요약}

본 과제에서는 HW\#1에 3D 장면 뷰어를 추가하였습니다.
이 뷰어에서는 사용자가 가상의 트랙볼을 드래깅하여 뷰를 회전시킬 수 있으며, 입력을 통해 translate, dolly in/out, zoom in/out을 수행할 수 있습니다.
또한 Show all 기능이 있어 장면 전체를 볼 수 있는 위치와 각도로 카메라를 자동으로 옮겨줍니다.
Seek 기능은 WebGL에서 depth buffer 접근에 어려움이 있어 구현하지 않았습니다.

\section{컴파일 및 실행}

WebGL을 지원하는 최신 브라우저와 npm이 시스템에 설치되어 있어야 합니다.
쉘에서 다음 명령어들을 수행하여 컴파일합니다.

\begin{enumerate}
  \item cd /path/to/src
  \item npm install
  \item webpack
\end{enumerate}

index.html을 브라우저에서 실행하면 됩니다.
또는 \url{http://csehydrogen.github.io/CG2016HW/hw2/}에서 실행해 볼 수 있습니다.
최신 Chrome에서 실행을 확인하였습니다.
조작법은 실행 후 좌측 상단에서 확인할 수 있습니다.

\section{구현}

\subsection{trackball}

먼저 창에서의 마우스 좌표 $(x', y')$를 trackball 위의 좌표 $(x, y, z)$로 변환해주어야 합니다.
창의 너비를 $w$, 높이를 $h$라 하고, trackball의 지름을 창의 높이에 맞춘다면, $x$와 $y$를 다음과 같이 먼저 구합니다.

\begin{align*}
  x &= \frac{x' - w / 2}{h / 2}\\
  y &= -\frac{y' - h / 2}{h / 2}
\end{align*}

만약 $x^2+y^2>1$이라면, $x^2+y^2=1$이 되도록 normalize 해줍니다.
그리고 $z=\sqrt{1-x^2-y^2}$입니다.

trackball 구현을 위해서는 마우스 버튼이 down 되는 순간의 좌표 $v_1$과 현재 드래그 중인 좌표 $v_2$가 필요합니다.
그리고 view matrix를 다음의 축과 각도로 회전변환 해줍니다.

\begin{align*}
  axis &= v_1 \times v_2\\
  angle &= atan{\frac{\lVert axis \rVert}{v_1 \cdot v_2}} + \pi
\end{align*}

\subsection{translate and dolly in/out}

view matrix가 trackball 중심 좌표와 각도를 결정하고, trackball 반지름 $r$을 알고 있다면 카메라의 위치 또한 결정할 수 있습니다.
그러므로 translate 동작은 view matrix를 translate 하였고, dolly in/out 동작은 $r$을 증감시켰습니다.

\subsection{zoom in/out}

Field of view 값을 증감시켰습니다.

\subsection{show all}

먼저, 카메라를 얼마나 뒤로 이동시켜야 하는지 결정해야 합니다.
편의상 장면을 원점에 놓인 반지름 $r$의 구로 가정하고 y축 방향 field of view를 $fovy$라 한다면, 카메라는 원점에서 최소 $minD$만큼 떨어져있어야 모든 장면을 볼 수 있고 그 거리는 다음과 같이 구합니다.

\begin{align*}
  minD = \frac{r}{\tan{\frac{fovy}{2}}}
\end{align*}

카메라 위치를 $p$, 각도를 $v$라고 했을 때, $\lVert p \rVert \geq minD$라면 카메라를 뒤로 이동할 필요는 없습니다.
그렇지 않다면 카메라를 $\lVert p + vt \rVert = minD$인 곳으로 이동시켜야하며, 이 때의 t는 다음 이차방정식을 풀어서 구합니다.

\begin{align*}
  (v\cdot v)t^2+2(v\cdot p)t+p\cdot p-minD^2=0
\end{align*}

위치가 결정된 후에는 카메라를 돌려 원점을 향하게 합니다.

\end{document}
