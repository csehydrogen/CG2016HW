\documentclass[12pt]{article}

\usepackage{kotex}
\usepackage{hyperref}

\begin{document}

\title{Programming Assignment \#1}
\author{김희훈}
\maketitle

\section{요약}

본 과제에서는 WebGL을 이용하여 움직이는 탱크 애니메이션을 구현하였습니다. 탱크는
몸통, 포탑, 포, 바퀴, 궤도로 구성되어 있고, 매트릭스 스택을 이용하여 계층 구조를
관리하였습니다. 키 입력을 받거나 자동모드를 설정하여 몸통을 제외한 부위를
움직이게 할 수 있습니다.

\section{컴파일 및 실행}

WebGL을 지원하는 최신 브라우저와 npm이 시스템에 설치되어 있어야 합니다.

\begin{enumerate}
  \item 첨부한 소스 파일을 저장합니다.
  \item 해당 폴더에서 npm install을 실행합니다.
  \item webpack을 실행합니다.
  \item 브라우저로 index.html을 실행합니다.
\end{enumerate}

또는 \url{http://csehydrogen.github.io/CG2016HW/hw1/}에서 실행해 볼 수 있습니다.
최신 Chrome에서 실행을 확인하였습니다.

\section{조작법}

실행 후, 좌측 상단에서 확인할 수 있습니다.

\section{모델}

E 50 Ausf. M이라는 디자인 상으로만 존재하는 탱크를 모델로 하였습니다. Blender를
이용하여 그린 후, 정점 좌표와 법선 벡터를 불러와서 사용하였습니다. 포와 바퀴는
원통, 궤도는 사각기둥, 포탑과 몸통은 다면체입니다.

\section{애니메이션}

몸통-포탑-포, 몸통-바퀴-궤도의 두 가지 3단 계층 구조를 가지고 있습니다. 상위
계층을 그린 후 모델뷰 매트릭스를 push하고, 하위 계층을 그린 후 모델뷰 매트릭스를
pop하는 식으로 모델들의 위치를 관리하였습니다.

\subsection{포탑}

몸통을 기준으로 좌표가 정해지며, y축 기준 360도 회전이 가능합니다.

\subsection{포}

포를 기준으로 좌표가 정해지며, x축 기준 $-20\sim+10$도 회전이 가능합니다. anchor가
포탑과의 연결부위에 있습니다.

\subsection{바퀴}

몸통을 기준으로 좌우 6개씩의 좌표가 정해집니다. x축 기준 360도 회전이
가능합니다.

\subsection{궤도}

좌우 34개씩의 얇은 사각기둥을 써서 궤도처럼 표현하였습니다. 이동경로가 바퀴를
기준으로 직선-반원-직선-반원인데, $[0,1)$을 이동경로상의 좌표에 대응시키는
함수를 만들고 $1/34$만큼의 간격으로 값을 대입하여 결과 좌표에 사각기둥을
배치하였습니다. 반원궤도상에 있는 사각기둥은 각도도 계산하여 회전시켰습니다.

\end{document}
